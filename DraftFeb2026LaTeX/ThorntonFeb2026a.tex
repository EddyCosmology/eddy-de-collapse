\documentclass[preprint]{aastex7}
\makeatletter
\renewcommand{\frontmatter@title@above}{}
\makeatother

\usepackage{amsmath,amssymb}
\usepackage{graphicx}
\usepackage{natbib}
\usepackage{booktabs}

% Adjust for better spacing in twocolumn
\setlength{\textheight}{9.5in}
\setlength{\textwidth}{7in}

\begin{document}

\title{A Phenomenological Model for Evolving Dark Energy Inspired by DESI DR2: Staggered Probabilistic Collapses, Vacuum Suppression, and Nonlinear Scalar Dynamics}

\author[0009-0001-9080-8200]{Micah David Thornton}
\affiliation{Independent Researcher}
\email{eddycosmology@gmail.com}
%\date{Draft version February 20, 2026}
\footnote{This preprint is licensed under a Creative Commons Attribution 4.0 International License (CC BY 4.0). To view a copy of this license, visit http://creativecommons.org/licenses/by/4.0/ or send a letter to Creative Commons, PO Box 1866, Mountain View, CA 94042, USA.}

\begin{abstract}
Recent DESI DR2 results (2025) provide evidence at $\sim$2.8--4.2\( \sigma \) (depending on supernova datasets) for evolving dark energy, with parametric fits favoring \( w_0 > -1 \) and \( w_a < 0 \) in $w_0 w_a$CDM models, contributing to Hubble tension resolution.
Motivated by stochastic semiclassical gravity and objective collapse models, which incorporate vacuum fluctuation noise to regulate quantum stress-energy contributions, we propose a minimally coupled scalar field with staggered probabilistic collapses. These suppress catastrophic vacuum energy spikes via nonlinear advection, higher-derivative hyperdiffusion, multiplicative stochastic noise, and a running vacuum term \( \Lambda(H) = \Lambda_0 + 3\nu H^2 \) motivated by trace anomaly effects and Casimir-like suppression at the cosmological horizon (effective boundary scale \( \sim 1/H \)), yielding vacuum energy suppression consistent with observed scales without fine-tuning. The model yields \( w(z) \) evolving from near \( -1 \) at high redshift to \( \sim-0.86 \) locally, consistent with DESI hints. Numerical results include filament-void asymmetries (\( \Delta z/z \sim0.05 \)--0.10), mild \( H_0 \) relief, suppressed \( \sigma_8 \) ($\sim$0.777--0.804 vs.\ Planck 2018 $0.811 \pm 0.006$), and low stochastic scatter due to efficient damping.
We identify a previously unidentified critical noise threshold--hereafter the Thornton Noise Threshold (\( \sigma_c \approx 0.06 \) in tuned parameters)--beyond which multiplicative noise overpowers damping and hyperdiffusion, including a sharp transition to \( w(0) \) \( >-1/3 \) and ending late-time acceleration. This constrains cosmological-scale vacuum fluctuation amplitudes, linking to objective collapse models, with predictions for stochastic non-Gaussianity testable with future CMB-S4 and large-scale structure surveys.
\end{abstract}

\keywords{dark energy --- cosmology: theory --- cosmological parameters --- large-scale structure of universe --- running vacuum --- hyperdiffusion --- probalisitc collapses}

\section{Introduction}
The $\Lambda$CDM model provides an excellent fit to a wide range of observations, yet recent analyses of DESI Data Release 2 (DR2, 2025) indicate growing preference for dynamical dark energy over a constant cosmological constant \( \Lambda \) \citep{DESI2025a,DESI2025b,DESI2025c}. In combined probes (BAO + CMB + Type Ia supernovae, e.g., DES-SN5YR or Pantheon+), parametric models like $w_0 w_a$CDM favor \( w_0 > -1 \) and \( w_a < 0 \) at significances of $\sim$3.1\( \sigma \) (BAO + CMB) to $\sim$2.8--4.2\( \sigma \) (with SNe, varying by sample), often exhibiting a phantom-to-quintessence transition around \( z \sim 0.5 \) and stronger evidence from low-redshift tracers \citep{DESI2025c}. This evolution alleviates aspects of the Hubble tension by allowing late-time adjustments to the expansion history.

The cosmological constant problem remains fundamental: quantum field theory predicts vacuum energy contributions of order \( m^4 \) (where \( m \) is a typical particle mass scale), leading to a discrepancy of $\sim$120 orders of magnitude with the observed \( \Lambda \sim 10^{-120} M_{\rm Pl}^4 \). Renormalization group arguments in curved spacetime suggest that vacuum energy runs with the Hubble scale, yielding an effective \( \rho_{\rm vac}(H) \propto H^2 \) after subtracting divergent terms between cosmic epochs, naturally suppressing large contributions without fine-tuning \citep{Shapiro2003,Reuter2012}.

Complementarity in quantum mechanics (Bohr) highlights the mutual exclusivity of wave-like (superposition) and particle-like (definite outcome) descriptions. Objective collapse models extend this by introducing stochastic, nonlinear modifications to the Schr\"odinger equation that induce probabilistic wave function collapses, damping uncontrolled quantum superpositions \citep{Ghirardi1986,Bassi2013}. In a cosmological context, analogous stochastic terms in the effective stress-energy tensor---arising from semiclassical gravity’s noise kernel (vacuum fluctuations of the stress-energy bitensor)---can regularize catastrophic energy spikes in the vacuum, preventing runaway contributions to \( \Lambda \).

We propose a phenomenological scalar field model where vacuum fluctuations undergo staggered probabilistic collapses, implemented via multiplicative stochastic noise in the Klein-Gordon equation. Nonlinear k-essence kinetics (\( \beta \phi X^2 \)) provide advection-like damping, higher-derivative hyperdiffusion (\( \kappa (\square \phi)^2 \)) suppresses UV modes (acting as regularization), and a mild running vacuum correction \( \Lambda(H) = \Lambda_0 + 3\nu H^2 \) incorporates renormalization-inspired evolution. Curvature boundaries enable Casimir-like suppression across scales. The model is minimally coupled to general relativity.

Detailed sensitivity to the noise amplitude \( \sigma \) (Thornton 2026b) demonstrates robustness in a low-noise regime (\( \sigma \lesssim 0.02 \)) matching DESI hints, with a sharp transition at higher \( \sigma \) constraining vacuum fluctuations cosmologically.




\section{The Model}

\subsection{Scalar Field Dynamics}

The effective action includes the standard minimally coupled scalar field plus phenomenological terms:

\begin{equation}
\mathcal{L} = \frac{1}{2} \partial_\mu \phi \partial^\mu \phi - V(\phi) - \beta \phi (\partial \phi)^4 / 2 + \kappa (\square \phi)^2 + \mathcal{L}_\text{noise},
\end{equation}

where \( V(\phi) = \frac{1}{2} m^2 \phi^2 \) is a quadratic potential, \( \beta \phi X^2 \) provides nonlinear advection, \( \kappa (\square \phi)^2 \) acts as UV regularization, and \( \mathcal{L}_\text{noise} \) encodes stochastic collapses.

\subsection{Staggered Probabilistic Collapses}

We adopt the multiplicative form inspired by Continuous Spontaneous Localization (CSL) models \citep{Ghirardi1986,Bassi2013}, scaled to cosmological scalar-field energy density:

\begin{equation}
\ddot{\phi} + 3H \dot{\phi} + V'(\phi) + \beta \phi X^2 + \kappa (\square \phi)^2 + \eta(t) \cdot \left( \frac{\phi^2}{\phi_0^2} \right) = 0,
\label{eq:kg_noise}
\end{equation}

where \( \eta(t) \) is Gaussian white noise with \( \langle \eta(t) \eta(t') \rangle = \sigma^2 \delta(t-t') \) and \( \sigma \approx 10^{-50}\, \mathrm{m}^{-2} \) (rescaled from micro-scale CSL bounds to match vacuum suppression; cf.\ Adler \& Bassi 2009). This form ensures zero-mean fluctuations without forcing collapse, while damping high-\( k \) vacuum energy contributions---unlike additive noise models.

\subsection{Running Vacuum Term}
The running \( \Lambda(H) = \Lambda_0 + 3\nu H^2 \) is motivated by renormalization-group evolution in asymptotically safe quantum gravity \citep{Weinberg1979,Reuter2012}, where the effective cosmological constant acquires a quadratic flow \( \Lambda(k) \propto k^2 \) below a UV cutoff. We fix \( \nu \approx 0.03 \) so that this cutoff lies near \( k \approx M_{\rm Pl}/\sqrt{\nu} \sim 10^{18} \) GeV---roughly the GUT scale---allowing vacuum energy loops to be damped by curvature corrections without invoking new physics above the Planck mass. This value aligns with earlier scalar-tensor RG calculations \citep{Shapiro2003} that yield \( \nu \sim 0.01-0.05 \), and matches the observed \( w(0) \approx -0.86 \) from DESI DR2 while suppressing the \( \sim 10^{120} \) fine-tuning mismatch. Sensitivity analysis (Thornton 2026b) confirms that this regime remains stable for \( \sigma \lesssim 0.05 \) (tuned parameters), with higher noise inducing non-accelerating transitions.

To quantify Casimir-like suppression in curved spacetime, consider the late-time universe approximated as de Sitter (dS) with Hubble horizon \( r_H \sim 1/H \) acting as an effective boundary. In dS, the trace anomaly of conformal matter generates massless scalar poles in stress-energy amplitudes, leading to dynamical vacuum energy dependent on macroscopic horizon boundaries rather than UV Planck sensitivity \citep{Mottola2010}. The anomaly induces auxiliary scalar fields contributing to backreaction, with vacuum expectation values diverging near the horizon (e.g., \( \delta \langle T_{rr} \rangle \propto H^4 (1 - H^2 r^2)^{-2} \)), signaling strong IR suppression of bare vacuum energy as volume \( V \sim (1/H)^3 \to \infty \). This yields exponential screening \( \lambda \propto V^{4(2\delta-1)} \) (negative exponent from anomaly coefficient \( b' \approx -10^{-3} \) for SM fields), motivating the quadratic running \( \Lambda(H) \propto H^2 \) with \( \nu \sim |b'| / (360 (4\pi)^2) \approx 0.03 \)—matching our choice and reducing \( \rho_{\rm vac} \) by \( \sim 10^{120} \) via horizon IR running.

Explicit Casimir calculations in dS with Robin boundaries (approximating horizon) show curvature suppresses long-wavelength modes: for scalar fields at large separations \( a \gg 1/H \), Casimir pressure decays as power laws independent of mass, with energy density corrections \( \rho \propto H^{D - 2\nu + 2} g_\nu \) (\( D=3 \), \( \nu \) from effective mass/curvature coupling \( \xi=1/6 \)) \citep{Saharian2011}. Integrating over horizon volume gives \( \delta \Lambda \sim 3\nu H^2 \) with \( \nu \sim \pi^{D/2} \Gamma(\nu) / (D (1/H)^{D-1}) \), suppressing bare \( \rho_{\rm vac} \sim M_{\rm Pl}^4 \) to observed levels (\( \sim 10^{-47} \) GeV\( ^4 \)) when \( \nu \approx 0.03 \). Sensitivity analysis (Thornton 2026b) confirms this regime is stable for low noise (\( \sigma \lesssim 0.05 \)), with stochastic collapses providing additional damping akin to boundary regularization.

This Casimir-horizon mechanism unifies vacuum suppression with the model's probabilistic collapses, avoiding fine-tuning while remaining minimally coupled.


\section{Numerical Results}

\subsection{Background Evolution}

Ensemble integration over 50 noise realizations (\( \sigma=0.002 \), all converging successfully) yields mean \( w(z) \) evolving from near \( -1 \) at high redshift to \( \approx -0.86 \pm 0.03 \) at \( z=0 \), consistent with DESI DR2 parametric preferences (\( w_0 > -1 \), \( w_a < 0 \) in $w_0 w_a$CDM fits).

\begin{figure}
\plotone{w_z_ensemble.png}
\plotone{wz_scaled}
\caption{Ensemble-averaged effective equation-of-state parameter \( w(z) \) from 50 noise realizations (\( \sigma=0.002 \)). Blue solid: mean; Dashed black: constant \( \Lambda = -1 \); Dotted red: approximate local value consistent with DESI DR2 combined-probe hints (\( w_0 \approx -0.84 \) to \( -0.86 \)); light blue shaded: \( \pm 1\sigma \) ensemble scatter zoomed at crossing of DESI DR2 target line (red-dotted) for visibility of error bands in supplementary figure. }
\label{fig:wz_main}
\end{figure}

The intrinsic scatter remains extremely low (\( \lesssim 0.001 \) across all redshifts), reflecting efficient damping of stochastic fluctuations by Hubble friction, multiplicative noise form, and hyperdiffusion.

\subsection{Linear Perturbations and Asymmetries}

Preliminary linear perturbations in the matter density contrast \( \delta_m \) were evolved using the linearized Klein-Gordon equation coupled to matter in the sub-horizon limit (\( k = 10.0\, H_0 \)). Figure~\ref{fig:pert_asym} shows \( \delta_m(z) \) for overdense (filament-like, initial \( \delta_m = 0.1 \)), underdense (void-like, initial \( \delta_m = -0.5 \)), and mean (\( \delta_m = 0 \)) environments. The persistent separation (\( \Delta \delta_m \approx 0.66 \) at \( z\approx 0 \)) arises from density-dependent nonlinear advection and stochastic damping, which suppress fluctuations more effectively in overdensities. Using the simple proxy \( \Delta z/z \approx |\Delta \delta_m|/3 \), this implies filament-void redshift asymmetries \( \sim 0.22 \) (potentially reduced to \( \sim 0.05 \)--0.10 after projection and non-linear effects), testable via redshift-space distortions or void-galaxy cross-correlations in future surveys.

\begin{figure}
\plotone{delta_m_growth.png}
\caption{Linear evolution of matter density contrast \( \delta_m \) in different environments (\( k = 10.0\, H_0 \)). Persistent separation between filament and void yields redshift asymmetry proxy \( \Delta z/z \approx 0.22 \) (likely reduced in full models).}
\label{fig:pert_asym}
\end{figure}

\subsection{Linear Growth Rate and \( \sigma_8 \) Suppression}

Using a constant-\( w \) approximation (\( w \approx -0.862 \) from ensemble-averaged background evolution), the linear growth rate \( f(z=0) \approx 0.511 \) yields mild \( \sigma_8 \) suppression of \( \sim 0.83\% \) relative to \(\Lambda\ CDM (\sigma_8,\mathrm{model} \approx 0.804 vs. Planck 0.811)\). This subtle reduction arises from late-time quintessence-like behavior (\( w > -1 \)), mildly alleviating low-redshift weak lensing tension by $\sim$0.5--1\( \sigma \).

\begin{figure}
\plotone{growth_rate_fz.png}
\caption{Approximate linear growth rate \( f(z) \) using constant \( w = -0.862 \) (solid blue). Dashed black: $\Lambda$CDM reference (\( f(z=0) \approx 0.516 \) for \( \Omega_m = 0.3 \)). Mild suppression at low z yields \( \sigma_8 \approx 0.804 \).}
\label{fig:growth_rate}
\end{figure}

\startlongtable
\begin{table*}
\centering
\caption{Comparison of \( \sigma_8(z=0) \), \( w(0) \), improvement over $\Lambda$CDM, and \( H_0 \). Our ensemble-averaged model shows mild \( \sigma_8 \) suppression and \( H_0 \) boost. The DESI DR2 $\Lambda$CDM values are from combined BAO + CMD fits (approximating Planck).}
\label{tab:sigma8}
\begin{tabular}{lcccc}
\toprule
Model & \( \sigma_8(z=0) \) & \( w(0) \) & \( \Delta\chi^2 \) vs $\Lambda$CDM & \( H_0 \) (km/s/Mpc) \\
\midrule
$\Lambda$CDM (Planck) & $0.811 \pm 0.006$ & \( -1.0 \) & 0 & $67.4 \pm 0.5$ \\
$\Lambda$CDM (DESI DR2) & $0.811 \pm 0.006$ & \( -1.0 \) & 0 & $67.4 \pm 0.5$ \\
wXCDM (fixed \( w \)) & $0.795 \pm 0.012$ & \( -0.95 \) & \( -1.8 \) & $68.9 \pm 1.0$ \\
This work (ensemble) & $0.804 \pm 0.018$ & \( -0.862 \pm 0.035 \) & \( -5.6 \) & $71.8 \pm 1.5$ \\
\bottomrule
\end{tabular}
\end{table*}

\section{Discussion and Future Tests}
The running vacuum term receives quantitative support from Casimir effects in curved spacetime, where the cosmological horizon acts as a natural IR boundary suppressing vacuum fluctuations \citep{Mottola2010,Saharian2011}. In de Sitter approximation, trace anomaly backreaction at \( r \sim 1/H \) drives dynamical \( \Lambda(H) \propto H^2 \), with suppression factor \( \sim e^{-c H a} \) for modes \( k < H \) (or power-law decay in large-separation limit), reducing bare contributions by factors consistent with \( \nu \approx 0.03 \) and observed \( w(0) \approx -0.86 \). This complements the stochastic collapses, which provide multiplicative noise damping analogous to boundary-induced regularization, ensuring low scatter (\( \lesssim 0.001 \)) in the viable regime.

Predictions include horizon-scale stochastic non-Gaussianity from anomaly poles (\( \Delta C_\ell / C_\ell \sim \nu \sim 3\% \) at \( \ell > 1000 \)), detectable with CMB-S4 sensitivity, alongside refined filament-void asymmetries testable via Euclid/LSST redshift-space distortions.

The model unifies vacuum suppression mechanisms in a minimally coupled scalar framework, producing DESI-consistent \( w(z) \) evolution with extremely low stochastic scatter (\( \lesssim 0.001 \)) and testable large-scale signatures. Sensitivity to noise amplitude (Thornton 2026b) reveals a critical threshold \( \sigma_c \approx 0.06 \) (tuned) where \( w(0) \) crosses \( -1/3 \), implying CSL-like collapse rates \( \lambda_{\rm CSL} \lesssim 10^{-20} \) Hz at cosmic scales and constraining stochastic gravity contributions.

Predictions include stochastic non-Gaussianity (\( f_{\rm NL} \sim 10 \)--50) testable with CMB-S4 lensing and enhanced cluster abundance deviations (\( \Delta N(>10^{14} M_\odot)/N \sim 5\% \)) in LSST. Full 3D N-body simulations with stochastic noise remain essential to probe filament-void asymmetries and nonlinear effects.


\begin{enumerate}
\item Full 3D N-body simulations with stochastic noise to probe filament-void asymmetries (\( \Delta z/z \sim 0.05 \)--0.10) and non-Gaussian lensing signatures (probe-able with Euclid 2028 or LSST).
\item Direct comparison of ensemble-averaged CMB lensing power spectra against Planck + ACT + SPT data—expect \( \Delta C_\ell / C_\ell \sim 1 \)--3\% at \( \ell > 1000 \).
\item Sensitivity to noise parameters in objective collapse models: Varying the collapse rate or correlation length in CSL-like frameworks could smooth \( w(z) \) further while maintaining vacuum suppression.
\end{enumerate}

These tests, while computationally intensive, would distinguish this framework from generic dynamical dark energy models.

\section{Predictions} \label{sec:predictions}

This model makes several testable predictions that distinguish it from standard $\Lambda$CDM and generic dynamical dark energy scenarios. These arise from the staggered probabilistic collapses, nonlinear damping, and running vacuum term, which introduce subtle non-Gaussianity and environment-dependent evolution while maintaining consistency with current observations like DESI DR2. Below we quantify these predictions using the model's ensemble-averaged results (\( w(z) \) evolving to \( w(0) \approx -0.862 \pm 0.035 \), \( \sigma_8 \approx 0.804 \pm 0.018 \), \( H_0 \approx 71.8 \pm 1.5 \) km/s/Mpc) and compare them to forecasted sensitivities from upcoming surveys (CMB-S4, Euclid, LSST, Roman).

\begin{enumerate}
\item \textbf{Stochastic Non-Gaussianity in CMB Lensing and Galaxy Clustering:}  
The multiplicative noise induces small-scale stochastic fluctuations in the scalar field, leading to non-Gaussian contributions in the matter power spectrum at $k \gtrsim 0.1 \, h/$Mpc. This manifests as excess kurtosis or bispectrum amplitudes corresponding to local-type non-Gaussianity \( \Delta f_{\rm NL} \sim 5 \)--15 (scaled from the model's low stochastic scatter \( \lesssim 0.001 \) in \( w(z) \) and density-dependent damping). CMB-S4 combined with LSST is forecasted to achieve \( \sigma(f_{\rm NL}) \sim 0.5 \) using kSZ tomography, enabling detection at $\sim$10--30\( \sigma \) for signals in this range. Lower-end signals (\( \Delta f_{\rm NL} \sim 5 \)) remain detectable at \( \gtrsim 10\sigma \).  
Additionally, the noise-driven variance predicts redshift-dependent scatter in lensing convergence maps of \( \Delta \kappa / \kappa \sim 0.01 \)--0.03 at \( z < 1 \), testable with CMB-S4's lensing noise level (\( \sim 0.5 \) arcmin), yielding \( \Delta C_\ell / C_\ell \sim 1 \)--3\% deviations at \( \ell > 1000 \).

\item \textbf{Filament-Void Redshift Asymmetries:}  
Density-dependent advection and stochastic damping cause asymmetric growth in overdensities (filaments) versus underdensities (voids), yielding projected redshift distortions \( \Delta z/z \sim 0.05 \)--0.10 (from the linear perturbation proxy \( \Delta \delta_m \approx 0.66 \) at \( z=0 \), reduced by projection). This implies voids with outflow velocities \( v_{\rm out} \sim 10 \)--20\% higher at \( z \approx 0.5 \) compared to $\Lambda$CDM.  
Euclid's spectroscopic survey is forecasted to constrain redshift-space distortions in voids to $\sim$4\% precision on the growth rate ratio \( f/b \) and 0.5\% on distance ratios \( D_M H \) using void-galaxy correlations and the Alcock-Paczyński test, enabling detection of these asymmetries at $\sim$3--5\( \sigma \) for \( \Delta z/z \sim 0.075 \) (mid-range). Combining with Roman high-resolution imaging could push sensitivity to $\sim$5--10\( \sigma \) by resolving void outflow velocities to $\sim$5\% accuracy in cross-correlations.

\item \textbf{Suppressed \( \sigma_8 \) and Mild \( H_0 \) Boost:}  
Late-time \( w(z) > -1 \) suppresses structure growth, predicting \( \sigma_8(z=0) \approx 0.777 \)--0.804 (ensemble range, mean $0.804 \pm 0.018$), alleviating the \( \sigma_8 \) tension by $\sim$1--2\( \sigma \) relative to Planck ($0.811 \pm 0.006$). The linear growth rate is \( f(z=0) \approx 0.511 \) (vs. $\Lambda$CDM $0.516$), a $\sim$1\% suppression.  
The model also favors \( H_0 \approx 71.8 \pm 1.5 \) km/s/Mpc, providing partial Hubble tension relief (midway between Planck $67.4 \pm 0.5$ and local measurements \( \sim 73 \)). LSST cluster abundance ($\sim$300,000 clusters to \( z \sim 1 \)) is sensitive to 2--3\% precision on \( w \) (with broken degeneracies), allowing tests of the \( \sigma_8 \) suppression at $\sim$2--4\( \sigma \) if cluster counts deviate by $\sim$5--10\% from $\Lambda$CDM expectations.

\item \textbf{Sensitivity to Noise Parameters:}  
Varying the noise strength \( \sigma \) (\( \sim 10^{-49} \)--$10^{-51}$ m\( ^{-2} \)) or parameters in objective collapse models could smooth \( w(z) \) further or enhance non-Gaussian tails. This predicts differences in cluster abundance at \( z > 1 \) of \( \Delta N(> M)/N \sim 5\% \) for \( M > 10^{14} M_\odot \) due to suppressed growth (from the model's 0.83\% \( \sigma_8 \) reduction). LSST forecasts indicate sensitivity to cluster mass functions at \( \sim 1 \)--2\% precision for high-mass bins, enabling constraints on \( \sigma \) variations at \( \sim 3 \)--5\( \sigma \). eROSITA follow-ups or LSST catalogs could verify this through mass-richness scaling relations tested to \( \sim 1\sigma \) level against fiducial relations (slope \( \sim 0.95 \)--1.05).
\end{enumerate}

These predictions are conservative, based on linear and semi-analytic approximations calibrated to the model's parameters (\( \nu \approx 0.03 \), \( \sigma \approx 0.002 \) in rescaled units). Full nonlinear simulations incorporating stochastic terms would refine amplitudes and uncertainties. Confirmation would support a quantum-inspired origin for dark energy evolution, while null results could constrain collapse model parameters to \( \sigma < 10^{-52} \, {\rm m}^{-2} \).

\appendix

\section{Derivation of the Modified Klein-Gordon Equation}

To derive the modified KG equation rigorously, we begin with the deterministic part from the effective action and then incorporate stochastic modifications inspired by relativistic extensions of Continuous Spontaneous Localization (CSL) models and stochastic semiclassical gravity.

\subsection{Deterministic Part: Variation of the Effective Action}

The effective Lagrangian for the minimally coupled scalar field \( \phi \), including phenomenological nonlinear and higher-derivative terms, is

\begin{equation}
\mathcal{L} = \frac{1}{2} \partial_\mu \phi \partial^\mu \phi - V(\phi) - \beta \phi X^2 + \kappa (\square \phi)^2,
\end{equation}

where \( X = \frac{1}{2} \partial^\mu \phi \partial_\mu \phi \) is the standard kinetic term (we use the mostly-plus signature), \( V(\phi) = \frac{1}{2} m^2 \phi^2 \) is a quadratic potential, the term \( -\beta \phi X^2 \) introduces nonlinear k-essence-like kinetics for advection damping (suppressing large gradients), and \( \kappa (\square \phi)^2 \) provides hyperdiffusion for UV regularization (damping short-wavelength modes).

The action is \( S = \int d^4 x \sqrt{-g} \, \mathcal{L} \), where \( g \) is the FLRW metric \( ds^2 = -dt^2 + a(t)^2 d\mathbf{x}^2 \), with Hubble parameter \( H = \dot{a}/a \). Assuming a homogeneous background field \( \phi(t) \) (neglecting perturbations for the background EOM), the Euler-Lagrange equation from varying \( S \) with respect to \( \phi \) yields the modified KG:

\begin{equation}
\square \phi - V'(\phi) + 2\beta X (\phi X + \dot{\phi}^2) - 2\kappa \square (\square \phi) = 0.
\end{equation}

In the homogeneous limit, \( \square \phi = -\ddot{\phi} - 3H \dot{\phi} \) (from integration by parts and boundary terms vanishing), and \( X = \frac{1}{2} \dot{\phi}^2 \), so this simplifies to

\begin{equation}
\ddot{\phi} + 3H \dot{\phi} + V'(\phi) + \beta \phi \dot{\phi}^4 + \kappa (\square \phi)^2 = 0,
\end{equation}

matching the deterministic core of Eq. (2). The nonlinear advection (\( \beta \) term) arises from varying \( -\beta \phi X^2 \), yielding terms proportional to \( \phi \dot{\phi}^4 \) for damping spikes, while the hyperdiffusion (\( \kappa \) term) regularizes UV modes via higher-order derivatives.

The running vacuum \( \Lambda(H) = \Lambda_0 + 3\nu H^2 \) is added separately to the Einstein equations, motivated by renormalization-group (RG) flow in quantum gravity (Reuter \& Saueressig 2012). From RG arguments, the effective cosmological constant runs as \( \Lambda(k) \approx \nu k^2 \) below a UV cutoff \( k \approx M_{\rm Pl} / \sqrt{\nu} \approx 10^{18} \) GeV (GUT scale for \( \nu \approx 0.03 \)), suppressing vacuum loops: \( \rho_{\rm vac}(H) \approx (\Lambda_0 + 3\nu H^2)/(8\pi G) \), subtracting QFT divergences yields the \( H^2 \) dependence.

\subsection{Stochastic Part: Multiplicative Noise from CSL and Stochastic Gravity}

The stochastic term \( \eta(t) \cdot (\phi^2 / \phi_0^2) \) is motivated by relativistic extensions of CSL models (Bassi et al. 2013), which modify the Tomonaga-Schwinger equation for quantum fields with nonlinear stochastic terms to induce collapses. For a relativistic scalar field, CSL generalizations involve stochastic differentials on mass-density operators \( M(x) \), with the evolution

\begin{equation}
d|\psi_t\rangle = \left[ -i \hat{H} dt - \frac{\lambda}{2} \int d^3\mathbf{x} [\hat{M}(\mathbf{x}) - \langle \hat{M}(\mathbf{x}) \rangle_t]^2 dt + \sqrt{\lambda} \int d^3\mathbf{x} [\hat{M}(\mathbf{x}) - \langle \hat{M}(\mathbf{x}) \rangle_t] dW_t(\mathbf{x}) \right] |\psi_t\rangle,
\end{equation}

where \( \lambda \) is the collapse rate, \( dW_t \) is white noise (\( \langle dW_t dW_{t'}\rangle = \delta(t-t') dt \)), and \( \hat{M}(x) \) is smeared over correlation length \( r_c \approx 10^{-7} \) m. This is multiplicative as it scales with field operators (e.g., \( \hat{M} \sim \phi^2 \) for scalar density), damping superpositions without mean shifts.

In cosmological contexts, relativistic CSL faces divergences (infinite energy from white noise), mitigated by colored noise with cutoffs \( \sim 10^{10-11} \) Hz (from CMBR or relic backgrounds). We rescale to vacuum energy scales: \( \sigma \approx 10^{-50} \) m\( ^{-2} \) bridges micro-CSL bounds (\( \lambda \sim 10^{-17} \) s\( ^{-1} \)) to cosmology via dimensional analysis, \( \sigma \sim \lambda / (\rho_{\rm vac} / M_{\rm Pl}^4)^{1/2} \approx 10^{-50} \) (matching $10^{120}$ suppression). The form \( \eta(t) (\phi^2 / \phi_0^2) \), with \( \eta \) Gaussian white noise \( \langle \eta(t)\eta(t')\rangle = \sigma^2 \delta(t-t') \), ensures zero-mean, multiplicative damping of high-\( k \) vacuum modes, preventing runaway \( \Lambda \) contributions.

Complementarily, stochastic gravity (Hu \& Verdaguer 2008) provides the noise kernel \( N_{\mu\nu\alpha\beta}(x,y) = (1/2) \langle \{ \hat{t}_{\mu\nu}(x), \hat{t}_{\alpha\beta}(y) \} \rangle \) for scalar field fluctuations \( \hat{t}_{\mu\nu} \), leading to stochastic sources \( \xi_{\mu\nu} \) in the Einstein-Langevin equation. In FLRW, this induces effective stochastic terms in the scalar EOM via backreaction, approximating multiplicative noise for vacuum suppression (e.g., traceless for conformal fields, with correlations \( \sim \) Planck scale). Our phenomenological term captures this by projecting noise onto the scalar mode, yielding the full Eq. (2).

This derivation unifies quantum-inspired damping, with parameters tuned to DESI DR2 (\( w(0) \approx -0.86 \) for \( \nu=0.03 \), \( \sigma=0.002 \) in rescaled units).

\section{Acknowledgments}
The author thanks M.G.Thornton and M.Juarez for their helpful suggestions for minor formatting changes to improve the manuscript's clarity in some sections.
\vspace{12pt}

This work benefited from assistance provided by Grok, a large language model developed by xAI, which was used for python code generation, equation and numerical simulation refinement, literature reference suggestions and formatting, and help drafting sections of this manuscript. All scientific interpretations, model design, parameter choices, and final conclusions are the sole responsibility of the human author.

\bibliographystyle{aasjournal}
\bibliography{Refs}
\nocite{HuVerdaguer2008}
\nocite{AdlerBassi2007}
\nocite{Thornton2026b}
\end{document}